\documentclass[english,,man,floatsintext]{apa6}
\usepackage{lmodern}
\usepackage{amssymb,amsmath}
\usepackage{ifxetex,ifluatex}
\usepackage{fixltx2e} % provides \textsubscript
\ifnum 0\ifxetex 1\fi\ifluatex 1\fi=0 % if pdftex
  \usepackage[T1]{fontenc}
  \usepackage[utf8]{inputenc}
\else % if luatex or xelatex
  \ifxetex
    \usepackage{mathspec}
  \else
    \usepackage{fontspec}
  \fi
  \defaultfontfeatures{Ligatures=TeX,Scale=MatchLowercase}
\fi
% use upquote if available, for straight quotes in verbatim environments
\IfFileExists{upquote.sty}{\usepackage{upquote}}{}
% use microtype if available
\IfFileExists{microtype.sty}{%
\usepackage[]{microtype}
\UseMicrotypeSet[protrusion]{basicmath} % disable protrusion for tt fonts
}{}
\PassOptionsToPackage{hyphens}{url} % url is loaded by hyperref
\usepackage[unicode=true]{hyperref}
\hypersetup{
            pdftitle={Non-word repetition in Yélî Dnye},
            pdfauthor={Alejandrina Cristia~\& Marisa Casillas},
            pdfkeywords={phonology, non-word repetition, development},
            pdfborder={0 0 0},
            breaklinks=true}
\urlstyle{same}  % don't use monospace font for urls
\ifnum 0\ifxetex 1\fi\ifluatex 1\fi=0 % if pdftex
  \usepackage[shorthands=off,main=english]{babel}
\else
  \usepackage{polyglossia}
  \setmainlanguage[]{english}
\fi
\usepackage{graphicx,grffile}
\makeatletter
\def\maxwidth{\ifdim\Gin@nat@width>\linewidth\linewidth\else\Gin@nat@width\fi}
\def\maxheight{\ifdim\Gin@nat@height>\textheight\textheight\else\Gin@nat@height\fi}
\makeatother
% Scale images if necessary, so that they will not overflow the page
% margins by default, and it is still possible to overwrite the defaults
% using explicit options in \includegraphics[width, height, ...]{}
\setkeys{Gin}{width=\maxwidth,height=\maxheight,keepaspectratio}
\IfFileExists{parskip.sty}{%
\usepackage{parskip}
}{% else
\setlength{\parindent}{0pt}
\setlength{\parskip}{6pt plus 2pt minus 1pt}
}
\setlength{\emergencystretch}{3em}  % prevent overfull lines
\providecommand{\tightlist}{%
  \setlength{\itemsep}{0pt}\setlength{\parskip}{0pt}}
\setcounter{secnumdepth}{0}
% Redefines (sub)paragraphs to behave more like sections
\ifx\paragraph\undefined\else
\let\oldparagraph\paragraph
\renewcommand{\paragraph}[1]{\oldparagraph{#1}\mbox{}}
\fi
\ifx\subparagraph\undefined\else
\let\oldsubparagraph\subparagraph
\renewcommand{\subparagraph}[1]{\oldsubparagraph{#1}\mbox{}}
\fi

% set default figure placement to htbp
\makeatletter
\def\fps@figure{htbp}
\makeatother

\usepackage{fontspec}

\setmainfont{Doulos SIL} % Set main font to Doulos SIL

\title{Non-word repetition in Yélî Dnye}
\author{Alejandrina Cristia\textsuperscript{1}~\& Marisa
Casillas\textsuperscript{2}}
\date{}

\authornote{All data are made available in a repository in the
Open Science Framework. AC acknowledges the support of the Agence
Nationale de la Recherche (ANR-17-CE28-0007 LangAge, ANR-16-DATA-0004,
ANR-14-CE30-0003, ANR-17-EURE-0017); and the J. S. McDonnell Foundation
Understanding Human Cognition Scholar Award.

Correspondence concerning this article should be addressed to
Alejandrina Cristia, 29, rue d'Ulm, 75005 Paris, France. E-mail:
\href{mailto:alecristia@gmail.com}{\nolinkurl{alecristia@gmail.com}}}

\abstract{
In nonword repetition (NWR) studies, participants are presented
auditorily with an item that is phonologically legal but lexically
meaningless in the local language. Accuracy is thought to reflect
long-term phonological knowledge as well as online phonological working
memory and flexible production patterns.In this study, we report on NWR
results among children learning Yêly Dnyé, an isolate spoken in Rossel
Island, PNG, with an unusually dense phonological inventory. This study
contributes to three lines of research. First, we document that non-word
items containing typologically rare sounds are repeated accurately less
often that non-words containing more common sounds. Second, we document
rather weak effects of item length, contributing to other research
suggesting that length effects may be language-specific. Third, we do
not find strong individual variation effects in this population,
contrary to previous results documenting strong age-related effects.
Together, these data provide a unique view of online phonological
processing in a seldom-studied language, and contribute to both
typological and language acquisition research.


}

\begin{document}
\maketitle

The Yélî sound system, much like its baroque grammatical system in
general {[}STEVE\_GRAMMAR\_REF{]}, is unlike any other in the region,
with 90 contrastive segments, including at least two contrasts not yet
been documented elsewhere (labial-coronal double-articulations with
dental vs.~post-alveolar coronal placement in both oral and nasal stops)
{[}STEVE\_GRAMMAR\_REF{]}. With only four primary places of articulation
(bilabial, alveolar, post-alveolar, and velar) and no voicing contrasts,
the phonological inventory is remarkably packed with acoustically
similar segments. The core oral stop set includes both singleton (/p/,
/t/, /ṭ/, and /k/) and doubly-articulated (/tp/, /ṭp/, /kp/) segments,
with full nasal equivalents (/m/, /n/, /ṇ/, /ŋ/, /nm/, /ṇm/, /ŋm/), and
with a substantial portion of these able to be pre-nasalized or nasally
released (/mp/, /nt/, /ṇṭ/, /ŋk/, /nmtp/, /ṇmṭp/, /ŋmkp/, /ṭṇ/, /kŋ/,
/ṭpṇm/, /kpŋm/). Finally, a number of these can also be labialized or
palatalized on the release, or both (/pʲ/, /tʃ/, /tpʲ/, /ṭʲ/, /ṭpʲ/,
/kʲ/, /kpʲ/, /pʲʷ/, /mbʲ/, /ndʒ/, /nmdbʲ/, /ṇḍʲ/, /ṇmḍbʲ/, /mbʷ/, /ŋɡʷ/,
/mbʲʷ/, /ṭṇʲ/, /ṭpṇmʲ/, /kŋʷ/, /mʲ/, /nʲ/, /nmʲ/, /ṇʲ/, /ṇmʲ/, /mʷ/,
/ŋʷ/, mʲʷ/). Among the remaining consonants are small group of oral
continuants (/w/, /j/, /ɣ/, /l/, /βʲ/, /lʲ/, /lβʲ/). Vowels in Yélî Dnye
may be oral or nasal, short or long. The 10 oral vowel qualities (/i/,
/ɯ/, /u/, /e/, /o/, /ə/, /ɛ/, /ɔ/, /æ/, /ɑ/) can be produced as short or
long vowels, with 7 of these able to appear as short and long nasal
vowels as well (/i/, /u/, /ə/, /ɛ/, /ɔ/, /æ/, /ɑ/). In total, Levinson
{[}STEVE\_GRAMMAR\_REF{]} then counts 90 distinctive segments in Yélî
Dnye (93 when including consonants that are extremely rarely used).

TESXwdfkh ɔ TESdfkjah grwerg /kʲ/, /kpʲ/, /pʲʷ/, /mbʲ/, /ndʒ/, /nmdbʲ/,
/ṇḍʲ/, /ṇmḍbʲ/, /mbʷ/, /ŋɡʷ/, /mbʲʷ/, /ṭṇʲ/, /ṭpṇmʲ/, /kŋʷ/, /mʲ/, /nʲ/,
/nmʲ/, /ṇʲ/, /ṇmʲ/, /mʷ/, /ŋʷ/, mʲʷ/). Among the remaining consonants
are small group of oral continuants (/w/, /j/, /ɣ/, /l/, /βʲ/, /lʲ/,
/lβʲ/) sdgfdfg

\setlength{\parindent}{-0.5in} \setlength{\leftskip}{0.5in}

\end{document}
